In Markosian \parencite*[]{Markosian_1997}, a paradox is presented.
An angel appears at a conference of philosophers and offers to answer only one question truthfully.
After careful consideration the philosophers ask \ref{Q4}:\footnotemark
\footnotetext{I follow the numbering of other's cited works throughout, and they mostly follow \parencite[]{Markosian_1997}.}

	\begin{principle}{Q4}\label{Q4}
	What is the ordered pair whose first member is the question that would be the best one for us to ask you, and whose second member is the answer to that question? 
	\end{principle}

The angel responds with \ref{A4}:

	\begin{principle}{A4}\label{A4}
	It is the ordered pair whose first member is the question you just asked me, and whose second member is this answer I am giving you. 
	\end{principle}

This is puzzling: \ref{Q4} seemed like an excellent question, but produces a useless answer.
The paradox is that \ref{Q4} appears to be at once the best question to ask, and \emph{not} the best question.

Assume that it is the best question: then there is no question better than it.
Assuming that questions are to be evaluated by how informative (or otherwise valuable) their answers are, \ref{Q4} surely can't be the best if its answer is \ref{A4}.
So \ref{Q4} cannot be the best question.

Assume that \ref{Q4} is not the best question: then there must be a question (call it Q4+) which is better than \ref{Q4} and will figure in the answer \ref{A4}.
Asking Q4+ will produce the answer A4+, whereas asking \ref{Q4} will produce <Q4+,A4+>.
The latter is more informative than the former: on top of the answer A4+, it tells us what question Q4+ is \emph{and} that it is the best question to ask.
Other things being equal, more informative questions are presumably better than less informative ones (call this the \textit{Principle of Informativeness}).
So asking \ref{Q4} is better after all than asking Q4+ because it is more informative.
\ref{Q4} will be better than any question Q4+, so \ref{Q4} now appears to be the best question.

In \parencite[]{Sider1997}, Sider solves the paradox: the presupposition that there is a best question has produced a contradiction (that \ref{Q4} is and is not the best question), so by \textit{reductio} there is no best question.
Instead of \ref{Q4}, Sider suggests \ref{Q5}:

	\begin{principle}{Q5}\label{Q5}
	What is an ordered pair consisting of one of the best questions we could ask and one of its answers?
	\end{principle}

Though it avoids the original problem, this produces the `real paradox of the question', (structurally similar to the original) that \ref{Q5} appears to be at once one of the best questions to ask, and \emph{not} one of the best.

Assume that \ref{Q5} is one of the best questions.
\ref{Q5} could be answered  with \ref{A5}:
	
	\begin{principle}{A5}\label{A5}
	<\ref{Q5},\ref{A5}>
	\end{principle}

This answer is as uninformative as \ref{A4} was.
Some questions---including first order questions---will only have possible answers better than \ref{A5}, so must be better than \ref{Q5}.
First order questions are not about other questions, but about ordinary matters like world hunger, so any first order question must tell us something more informative than \ref{A5}.
So \ref{Q5} can't be one of the best questions.
\parencite[4]{Sider1997}

Assume that \ref{Q5} is not one of the best questions: then there must be a question (call it Q5+) which is better than \ref{Q5} and will figure in the answer \ref{A5}.
Asking Q5+ will produce the answer A5+.
Now that it is not one of the best questions, asking \ref{Q5} will produce <Q5+,A5+>.
The latter is more informative than the former: on top of the answer A5+, it tells us what Q5+ is \emph{and} that it is the one of the best questions to ask.
Asking Q5+ \emph{just} tells us A5+.
Then by the principle of informativeness, \ref{Q5} must be better than Q5+ (which is one of the best questions), so \ref{Q5} must be one of the best questions.

This is Sider's `real paradox of the question': `\ref{Q5} cannot be consistently supposed to be one of the best questions to ask, but neither can it be supposed to not be one of the best questions'.

This paradox no longer supposes that there is \emph{a} best question but is does assume that there are \emph{some} best questions.
So in keeping with the solution to the first paradox, we could solve the real paradox by denying that there any best questions.
Wasserman and Whitcomb mention briefly that they support this but do not elaborate. 
\parencite[151]{Wasserman_2011}
Sider writes `it is hard to believe that we could be forced to accept such a conclusion by a priori means'.
\parencite[4]{Sider1997}

Instead, Wasserman and Whitcomb offer a different solution, denying that \ref{Q5} is problematic and thus rejecting the first horn of the dilemma.

They argue that the `danger' with asking \ref{Q5} is not unique to it, but shared by all (or many) questions. 
Though the angel could give a useless answer to \ref{Q5} (like \ref{A5}), it could just as well give a useless answer to `Who is the author of Huckleberry Finn?' (like `the author or Huckleberry Finn' or `my favourite author') or any other question.
Call the potential for such a useless answer the \textit{problem of pedantry} because these answers would be pedantic in normal conversation.
If all questions suffer from this problem then although \ref{Q5} does, it might be no worse than than them and still be one of the best questions.

In so far as all questions can be answered in an unhelpful way, clearly \ref{Q5} is comparable with many other questions.
After all, `my favourite author' and \ref{A5} \emph{are} both unhelpful answers.
But Wasserman and Whitcomb's argument requires more: that the disadvantage of asking a first order question (derived from its problem of pedantry) be equal to that of asking Q5.

Only if this is true would \ref{Q5} be no worse than a first order question.

The problem with Q5 is that one of its potential answers is <\ref{Q5},\ref{A5}>.
Call this the \textit{problem of self-answering} because here \ref{Q5} figures in it's own answer.

So Wasserman and Whitcomb's solution requires that the case of the problem of self answering in \ref{Q5} is no worse than a case of the problem of pedantry in a first order question.

This is not so.
\ref{Q5*} could be answered unhelpfully (and so suffers from the problem of pedantry), but cannot receive an answer which includes itself (and \emph{does not} suffer from the problem self-answering)

	\begin{principle}{Q5*}\label{Q5*}
	What is an ordered pair consisting of one of the best questions we could ask, (apart from this one and Q5) and one of its answers?
	\end{principle}

\ref{Q5*} shows that the two problems (or two instances of the same problem) can come apart.
If the two problems are distinct, then \ref{Q5} has at least one problem which most questions do not.
Even if the problem of self-answering is a special case of the problem of pedantry \ref{Q5*} at least shows that \ref{Q5} has two different instances of the problem.

Either way the risk of asking \ref{Q5} is greater than that of asking a first order question, so \ref{Q5} cannot be one of the best questions and Wasserman and Whitcomb's solution is unsuccessful.

The other natural solution is to follow the same reasoning as with the original paradox and deny that there are any best questions.
If the assumption that there is a group of best questions produces a paradox, then a methodologically attractive solution is to treat the paradox as a \textit{reductio} and reject the assumption.
The reasoning is clear but the conclusion seems very counter-intuitive.

I think that although the reasoning above shows us that in the strictest sense there are no best questions, our intuitions can still be accommodated.

A working definition of best: in the argument presented above, a question is precluded from being the best (or among the best) if another question is better than it.
So for some question Q to be one of the best questions just is for there to be no better questions than Q.
Notice that this allows for a group of best questions, if they are all of equal value, and there are no other questions of greater value.

Hence, that there are no best questions just is that for every question there is at least one better question.
Thus there is an infinite series of progressively better questions.

This in itself is not so surprising.
Any question can itself be the subject of a further question.
If a series of these questions were constructed recursively then each question would be unique and each would be the subject of a further question so the series would be infinite.
Considering the sort of higher-order questions discussed above (which ask for both the answers \emph{to}, an information \emph{about} lower order questions), we can construct an infinite series of progressively better questions.

Take any first order question QO1.
From QO1 we can then construct a second order question \ref{QO2}:

	\begin{principle}{QO2}\label{QO2}
	What is the ordered pair <A1,W1> where A1 is the answer to QO1 and W1 is why A1 is the case?
	\end{principle}

The answer to \ref{QO2} will provide all the information the answer to QO1 would have (that is, it will tell us A1) \emph{and} will tell us why that is the case.
On having our factual questions answered, we often ask `why' of the answer, so W1 will be valuable enough that \ref{QO2} is more informative than QO1, and therefore better.
We could then construct an even better question:

	\begin{principle}{QO3}\label{QO3}
	What is the ordered pair <A2,W2> where A2 is the answer to QO2 and W2 is why A2 is the case?
	\end{principle}

Similar reasoning shows that \ref{QO3} is better than \ref{QO2}, QO4 is better than QO3, \ldots\ etc..
The paradox of the question aside, an infinite series of progressively better questions is unsurprising.
This gives us independent reason to think that there really are no best questions---how could there be if there are infinite series like this?

Scott \& Scott make a similar point concluding `the only possibility for the best possible question is a question whose answer tells us everything'.
\parencite[332]{Scott_1999}
Although they suggest such a question, their difficulty in reforming it to be practical brings them the same conclusion as me.
I will develop some further consequences of this view, and show that---on a less strict notion of `best'---it is compatible with our intuitions.

Whatever value is used to measure one question against another, I think it is very likely that it will have an upper limit for humans.
Take benefit (where Q is better than Q' iff asking Q produces more benefit).
Human lives---and eventually human history---are finite, as are the means and resources for improving them and the information (and questions for that information) which informs these means.
Although any one question might be more beneficial than another, the questions can't just go on becoming more and more beneficial forever: there is only so much that could possibly be done to benefit humanity.
The same stands for any other value used to compare questions.

Since there are an infinite number of questions, the difference in value between one question and the next must decrease as the questions become more and more valuable.

For each actual question in the series, there will an infinite number of questions with greater values than it (further on in the series), which take up the `space' between the question and the maximum value.
With each subsequent question the `space' becomes smaller by some difference in values d (the difference between the value of one question and the next).
But only a finite multiple of d (and so a finite number of further questions) can be added on top of any value without hitting the maximum.
No question \emph{can} have the maximum value, because such a question would be truly the best.
So to accommodate more questions d will have to decrease as the number of questions increases (as the available `space' for questions decreases).
% The smaller d is, the more questions will fit into the same space, so at some point d must decrease to make room for more (as there are infinite questions).
There are infinite questions, so d will have to decrease infinitely.
So as the series continues through better and better questions, the difference between the questions must decrease.

As a practical example, consider the series QO1, \ref{QO2}, \ref{QO3}, \ldots\ .
Each question tells us what its predecessor would have, and why that is the case.
For early questions each reason Wn will be of significant value, and might differ significantly from Wn-1.
But as the series progresses, the reasons are likely to become more difficult for humans to grasp, and more similar to each other.
If QO1 is a practical question like `When is the best time to check your car's oil?', then the reason W100 might just be an appeal to a physical law, and W101 a similar appeal to a more general law.
W100 and W101 will be very similar, the difference in value between them (and thus the difference in value between QO100 and QO101) will be much smaller than an earlier difference (e.g.\ between QO1 and \ref{QO2}).

There must be a group of questions so close to the maximum and to each other (as d decreases to be negligibly small) that the difference is immaterial for a human actually asking a question of the angel.
The precision of human perception of qualities like benefit and informativeness is limited (even if it is very fine in some individuals), and eventually the difference d will become so small that it is not noticed.
Although the questions above this point do keep on getting better and better, the actual outcomes of asking them are all so close to being on a par that for a human they may as well all be the best.

These questions are the best in two senses.
They are the \emph{best}, in that for a normal person they are indistinguishable from questions which provide the maximum possible value.
\emph{They} are the best in that there will be many such questions (indeed, an infinite series of them), so the solution to the original paradox remains intact.

This notion of `best' is actually much closer to the notion we employ in everyday use in sentences like `she is one of the best in her class'. (and I believe, closer to the intended sense in Sider's \ref{Q5})
Imagine a runner who places in the top ten in their race in every Olympic games from their first entry until their retirement, but never comes first.
Over their whole career there will always be at least one person better than them, so they are not \emph{the best} in the strict sense above.
But we would not hesitate to call them \emph{one of the best} runners.
This is a notion of best we often employ, and it is very similar to my solution to the paradox: though the runner is never strictly \emph{the} best, they are still so good that for you or me they may as well be.
We express this with phrases like `one of the best'.
The same is true of the questions.

This model answers the real paradox of the question, by denying the presupposition in the question itself.
However, it also does justice to our intuitions that some questions are better than others, and that there is a question (or group of questions) we \emph{should} ask the angel.
Moreover, there is an infinite group of these `best' questions, so should the angel ever return, we shall not have to start our search for the next best question all over again.
Our next question may even be better.
