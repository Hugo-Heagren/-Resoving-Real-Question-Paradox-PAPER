This paper defends a solution to the \emph{real} paradox of the question.
Sider solved the original paradox of the question by accepting that there is no best question to ask an omniscient angel and suggested instead trying to ask one of the best questions.
This approach produces a similar paradox: the `real paradox of the question'.
Wasserman and Whitcomb recognise that a similar strategy seems unpalatable---surely we can't conclude there are no best questions---and offer an alternative solution accordingly.
I first show that Wasserman and Whitcomb's solution is unsuccessful and stems from a misunderstanding.
I then argue  that instead the paradox shows we should accept there are no best questions.
This is unintuitive though, so finally I give an account of the questions we could ask which accommodates this fact and our intuitions.
This solves the paradox.
